\documentclass{article}

\usepackage{amsmath}
\usepackage{amsfonts}

\nofiles

\title{Molecular Dynamics Project}
\author{Matt Nappo}
\date{\today}

\begin{document}

\section{Math Review}

A vector-valued function is a function that takes in n input variables and puts those input variables into component scalar functions. Example: $$f(t, v) = f_1(t, v)\hat x + f_2(t, v)\hat y + f_3(t, v)\hat z$$ There can be as many t, v as you want. This is basically just a vector field (they're basically the same thing for me at least), a (3D) plane where every point is a vector (assuming no domain restrictions). When you plug in real actual numbers into the vector function (co-domain), the values are put into the scalar functions and each scalar function computation is separate. You end up with a vector.

A scalar-valued function is something just like $$f(x_1, x_2, x_3) = x_1^2 + x_2^2 + x_3^2$$ where the output of the function is just a scalar.

You take the gradient of a scalar-valued function. It doesn't really make sense to take the gradient of a vector-valued function. The gradient is just all of the partial derivatives of the scalar function put into a vector. The gradient of a scalar-valued function will give you a vector-valued function. So, the gradient at a single point is just like evaluating a regular vector-valued function which I described above: you just plug the values into each component scalar function and you end up with a vector. However, in this case, because the components of the vector function are the partial derivatives of some other function, the vector that you end up with when you evaluate the gradient at a certain point is the slope of the tangent line in the x direction, y direction, and z direction. Pretty cool, huh!

\section {Computational Differentiation}

\subsection {Numerical Differentiation}
Bad, slow. Numerical differentiation is when you just $$\frac{f(x+h)-f(x)}{h}$$ and choose a really small $h$. Limited by the precision of $h$.

\subsection {Algorithmic/Automatic Differentiation}
Forward, backward, and midpoint method.
Basically the idea that you multiply some matrix $D$ by a vector which represents the function you are differentiating. Each row of the vector is the
value of the function at that point. A greater $N$ means more precision because now you're splitting up the range more and getting more and more accurate
values of the function at each point. 
$$
\dfrac{df}{dx} = 
\begin{bmatrix}
	d_{00} & \cdots & d_{0N} \\
	\vdots & \ddots & \vdots \\
	d_{N0} & \cdots & d_{NN}
\end{bmatrix}
\begin{bmatrix}
	f(x_0) \\
	\vdots \\
	f(x_N)
\end{bmatrix}
\text{ as } N \rightarrow \infty
$$

\subsection {An Extension to Multiple Dimensions}
So, now we have this new structure $\mathbf{D}$ that lets us differentiate a function. Just to be super clear, $$f' = \mathbf{D}f = \dfrac{df}{dx}$$
Surely, we can define $f: \mathbb{R}^n \rightarrow \mathbb{R}$. Let's use an example in $\mathbb{R}^2$ and define $f(t, v)$.
$$\mathbf{D}_tf = \dfrac{\partial f(t, v)}{\partial t} = f_t \text{ and } \mathbf{D}_vf = \dfrac{\partial f(t, v)}{\partial v} = f_v$$
Therefore,
$$
\nabla f = \begin{bmatrix}
	\mathbf{D}_tf \\
	\mathbf{D}_vf
\end{bmatrix}
$$

But what does this really mean? Let's be more explicit with an example where $N = 4$.

$$
\mathbf{D}f = \nabla f = \begin{bmatrix}
	-1 &  1 &  0 & 0 \\
	 0 & -1 &  1 & 0 \\
	 0 &  0 & -1 & 1 \\
	 0 &  0 & -1 & 1
\end{bmatrix}
\begin{bmatrix}
	f(t_0, v_0) \\
	f(t_1, v_1) \\
	f(t_2, v_2) \\
	f(t_3, v_3)
\end{bmatrix}
$$
Or is it
$$
\nabla f(t, v) = \begin{bmatrix}
	-1 &  1 &  0 & 0 \\
	 0 & -1 &  1 & 0 \\
	 0 &  0 & -1 & 1 \\
	 0 &  0 & -1 & 1
\end{bmatrix}
\begin{bmatrix}
	f_t(t, v)\hat \imath + f_v(t, v)\hat \jmath \\
	f_t(t, v)\hat \imath + f_v(t, v)\hat \jmath \\
	f_t(t, v)\hat \imath + f_v(t, v)\hat \jmath \\
	f_t(t, v)\hat \imath + f_v(t, v)\hat \jmath
\end{bmatrix}
$$

Vector within a vector??
Now I'm confused. That makes no sense. 
Also: same range for $t$ and $v$?

\end{document}

