\documentclass{article}

\usepackage{amsmath}
\usepackage{amsfonts}
\usepackage{indentfirst}
\usepackage[margin=0.5in]{geometry}

\nofiles 
\title{Kinetic Theory}
\author{Matt Nappo}
\date{\today}

\begin{document}

\maketitle

\section{Basic Definitions}
$$\vec{v} = \left< v_x, v_y \right> \,\text{(velocity)}$$

$$\vec{v} \cdot \vec{v} = v^2 = v_xv_x + v_yv_y \,\text{(velocity squared)}$$

$$\vec{p} = m\vec{v} \,\text{(momentum)}$$

$$K = \frac{1}{2}mv^2 = \frac{p^2}{2m} \,\text{(kinetic energy of a particle)}$$

$$ E = kT\,\text{(energy,}\, k\, \text{is the Boltzmann constant)}$$

%$$<E> = kT \,\text{(average energy of the system)}$$
%$$K_{tot} = \sum_i^N \frac{1}{2}mv_i^2 \,\text{(total kinetic energy of the system)}$$

\section{Derivations}
Because this is a 2D ideal gas moving in a natural environment, we can assume that $\overline{v_x} = \overline{v_y}$ because the gas has no external force applied to it; it is merely moving naturally.
Because the gas is ideal, $$\Delta p = p_{i,x} - p_{f,x} = p_x - (-p_x) = 2p_x$$ when a particle of the gas hits a wall. This is because the speed remains the same and because the direction of the perpendicular component changes upon a collision with a wall.  Therefore, the time $\Delta t$ between each particle's collisions with a wall is $\frac{2L}{|v_x|}$ assuming that the area of the 2D space is $L^2$.
This change in momentum upon a collision is a force. The force of a singular particle colliding with a wall in the x direction is therefore
$$F_x = \frac{\Delta p}{\Delta t} = \frac{2mv_xv_x}{2L} = \frac{mv_x^2}{L}$$

The force exerted by all of the particles colliding with a single wall in the x direction is the sum of all the forces acting on that wall. So,
$$F_{tot\,x} = \frac{\sum_i^N \Delta p_i}{\Delta t} = \frac{\sum_i^N 2mv_{x,i}}{\Delta t} = \frac{2m \sum_i^N v_{x,i}}{\Delta t} = \frac{2m \sum_i^N v_{x,i}}{\frac{2L}{v_x}} = \frac{m \sum_i^N v_{x,i}^2}{L} = \frac{Nm\overline{v_x^2}}{L}$$

Because, as stated earlier, $\overline{v_x} = \overline{v_y}$, it follows that $\overline{v_x^2} = \overline{v_y^2}$. So, $$\overline{v^2} = \overline v \cdot \overline v = 2\overline{v_x^2}\,\text{and}\, \overline{v_x^2} = \frac{\overline{v^2}}{2}$$
Replacing $\overline{v_x^2}$ in the force equation above gives $F = \frac{Nm\overline{v^2}}{2L}$ and because this pressure is being exerted over a wall of length $L$,
$$P = \frac{F}{L} = \frac{\frac{Nm\overline{v^2}}{2L}}{L} = \frac{Nm\overline{v^2}}{2L^2}$$

Because the gas is ideal and the only forces in the entire container are from particles colliding with the wall, the gas particles don't hold any potential energy. This means that all of the energy for the entire system is stored as kinetic energy, meaning $$\overline K = N\frac{1}{2} m\overline{v^2} = \overline E = kT$$ because $K = \frac{1}{2} mv^2$ So,
$$ PA = \frac{1}{2} \overline K = NkT$$
This is just $PV=nRT$ but with different units and in 2D with area instead of volume.

\end{document}

