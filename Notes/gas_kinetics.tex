\documentclass{article}

\usepackage{amsmath}
\usepackage{amsfonts}
\usepackage{indentfirst}

\nofiles

\title{Kinetic Theory}
\author{Matt Nappo}
\date{\today}

\begin{document}

\maketitle

\section{Basic Definitions}
$$\vec{v} = \left< v_x, v_y \right> \,\text{(velocity)}$$

$$\vec{v} \cdot \vec{v} = v^2 = v_xv_x + v_yv_y \,\text{(velocity squared)}$$

$$\vec{p} = m\vec{v} \,\text{(momentum)}$$

$$K = \frac{1}{2}mv^2 = \frac{p^2}{2m} \,\text{(kinetic energy of a particle)}$$

%$$<E> = kT \,\text{(average energy of the system)}$$
%$$K_{tot} = \sum_i^N \frac{1}{2}mv_i^2 \,\text{(total kinetic energy of the system)}$$

\section{Derivations}
Because this is a 2D ideal gas moving in a natural environment, we can assume that $|v_x| = |v_y|$ because the gas has no external force applied to it; it is merely moving naturally.
Because the gas is ideal, $$\Delta p = p_{i,x} - p_{f,x} = p_x - (-p_x) = 2p_x$$ when a particle of the gas hits a wall. This is because the speed remains the same and because the direction of the perpendicular component changes upon a collision with a wall.
Therefore, the time $\Delta t$ between each particle's collisions with a wall is $\frac{2L}{|v_x|}$ assuming that the area of the 2D space is $L^2$.
This change in momentum upon a collision is a force. The force of a singular particle colliding with a wall is therefore given by
$$F = \frac{\Delta p}{\Delta t} = \frac{2mv_xv_x}{2L} = \frac{mv_x^2}{L}$$

The force exerted by all of the particles colliding with a single wall is
$$F = \frac{1}{L} \sum_i^N \frac{1}{2}mv_{i,x}^2 = \frac{m}{2L} \sum_i^N v_{i,x}^2$$
because
This can also be written as
$$F = \frac{Nm\overline{v_x^2}}{L}\, \text{because}\, \overline{v_x^2} = \frac{1}{N}\sum_i^N v_{i,x}^2$$

\end{document}

